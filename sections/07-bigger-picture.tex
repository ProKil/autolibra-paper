\documentclass[../main.tex]{subfiles}

\begin{document}
	\section{The Bigger Picture}

	Humans learn not only from their own experience, but crucially from collective
	knowledge transmitted through social interaction \citep{tomasello1993cultural}. This
	capacity for social learning and teaching enables knowledge accumulation across
	generations, allowing us to build upon rather than reinvent skills and artifacts
	\citep{humphrey1976social}. At the core of this capability lies a distinctive
	form of intelligence that sets humans apart from species lacking cumulative culture:
	social intelligence. A fundamental question emerges: how can we leverage this social
	intelligence to enable AI agents to learn from humans and potentially teach them
	in similar ways?

	Current training paradigms for language agents predominantly focus on behavior
	cloning or reinforcement learning from self-generated experience. However, behavior
	cloning has inherent limitations for skill acquisition. First, it requires teachers
	or demonstrators to operate in the same environment as the learning agent, constraining
	the scalability of knowledge transfer. Second, agents learning from one or a few
	demonstrations may develop misunderstandings about what constitutes appropriate
	behavior. In such cases, explicit verbal feedback from teachers becomes crucial for
	establishing the boundaries of acceptable behaviors and clarifying underlying principles
	that cannot be easily inferred from demonstrations alone.

	AutoLibra offers a computational approach to address these limitations by incorporating
	the verbal feedback that humans naturally exchange in daily interactions to systematically
	improve agent behavior. Looking forward, the broader vision is to deploy AI agents
	in real-world settings where they can learn from diverse human feedback at scale.
	This approach could enable agents to acquire skills from the collective experience
	of all users, forming both general competencies and specialized skill sets at
	\emph{personal} and \emph{organizational} levels, mirroring how human communities
	accumulate and share knowledge.

	Going one step further, such agents could serve as intermediaries for accelerating
	skill and culture transmission between humans themselves---distilling expertise from
	proficient users into interpretable metrics and actionable feedback that can guide
	novices, thereby compressing traditional learning curves and democratizing access
	to specialized knowledge.  
	
\end{document}