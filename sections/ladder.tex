\usepackage{fontspec}
\usepackage{emoji}
\usepackage[table,xcdraw]{xcolor}
\section{The ladder \protect\includegraphics[height=1em]{figs/ladder.png}: Agent improvement with AutoLibra}


Methods
Improvements
Prompt
Failed attempts
Scores
Defeat null hypothesis of:
- Goal score alone being sufficient
- Human-derived metrics alone being sufficient

(~2 pages total incl. tables, shortened prompts)

Turn 0:

Metrics obtained are:
\begin{itemize}
    \item Win Condition Recognition (WCR)
    \item Rule Modification for Obstacle Management (RMO)
    \item Direct Navigation Efficiency (DNE)
    \item Context-Sensitive Decision Making (CSDM)
\end{itemize}
Win condition recognition prequisite to all other metrics, turn 1 improvements target this metric specifically.

Turn 0 trajectory appears stochastic, ∴ impossible for human to isolate model behaviors for targeted improvement; very low goal score/random trajectory makes using goal score as an agent improvement metric not useful in this case.

Turn 1 improvement and motivation:
\begin{enumerate}
    \item Few-shot prompting, targets WCR by giving model example of identifying win condition from native env observations
    \item Subtask-level planning guidance, target improvement of CSDM and WCR by mapping environment observations to goals, which model could not do in turn 0
\end{enumerate}

Turn 1:

\begin{itemize}
    \item Improvement in WCR vs Turn 0, indicating T1 improvements worked
    \item Slight improvement in RMO, DNE, CSDM but not statistically significant
    \item Overall improvement in BALROG score because agent seeks objectives
    \item Qualitatively, agent now targets specific blocks instead of randomly walking - but also now gets stuck when objective complex or multi-step
\end{itemize}

Metrics obtained are:
\begin{itemize}
    \item Win Rule Construction (WR)
    \item Selective Interaction with Relevant Objects (SI)
    \item Rule Manipulation and Execution (RME)
\end{itemize}

Turn 2 unsuccessful improvements (Decreased performance)
\begin{enumerate}
    \item 1-step location and action history as part of observation
    \item 5-step location and action history as part of observation
    \item Movement templates (given a start location and destination, what movements are necessary to navigate without colliding with obstacles)
\end{enumerate}

Turn 2 improvement and motivation:
\begin{enumerate}
    \item Augment subtask-level planning instructions with positional instructions, targets RME and RMO
    \item Meta-prompting by providing condition-based task identification heuristic, targets SI, WCR, and WR
    \item Augment few-shot example with movement guidance
\end{enumerate}


Turn 2:


\begin{itemize}
    \item Substantial improvement in WCR due to more detailed, condition-based task identification heuristic, corresponding increase in other metrics supports idea that WCR is prerequisite metric to other reasoning metrics
    \item Strong increase in RME and RMO, indicating improved agent reasoning performance induced by more detailed subtask-level planning  
    \item SI no improvement, agent still misunderstands map directions and confuses rule blocks with objects
    \item Qualitatively, agent now recognizes when wall needs to be broken and breaks wall rule; still cannot assemble rules to win; still gets stuck on immovable blocks and walls
\end{itemize}

Metrics obtained are:
\begin{itemize}
    \item Subtask Coordination and Overall Task Planning (SC)
\end{itemize}

Turn 3 improvement and motivation:
\begin{enumerate}
    \item Format observations as absolute position (as opposed to relative) and 
    \item Chain-of-thought reasoning to assemble goal agenda from
    \item Many-shot example of goal reasoning
\end{enumerate}


Turn 3: